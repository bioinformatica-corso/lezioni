\documentclass[12pt,aspectratio=169]{beamer}
\usepackage{pxfonts}

\usepackage{fancyvrb}
\fvset{%frame=single,
commandchars=\\\{\},
%framesep=1mm,
fontfamily=helvetica,
fontsize=\normalsize
}
%\usepackage[bitstream-charter]{mathdesign}
\usepackage{listings}
\lstset{commentstyle=\color{orange}, keywordstyle=\color{yellow} }

\usepackage{newpxtext}
\usepackage[utf8]{eulerpx}

\usetheme{Boadilla}
\useoutertheme{split}
\usecolortheme{albatross}


\setbeamertemplate{blocks}[rounded][shadow=false]
\setbeamertemplate{navigation symbols}{}
\beamerdefaultoverlayspecification{<+->}


\setbeamercolor*{structure}{fg=green!75!black,bg=blue!70!white}
\setbeamercolor*{normal text}{fg=green!65!black,bg=blue!80!black}
\setbeamercolor{palette primary}{use={structure,normal text},fg=green,bg=structure.bg!75!black}
\setbeamercolor{palette secondary}{use={structure,normal text},fg=structure.fg,bg=structure.bg!60!black}
\setbeamercolor{palette tertiary}{use={structure,normal text},fg=structure.fg,bg=structure.bg!45!black}
\setbeamercolor{palette quaternary}{use={structure,normal text},fg=green,bg=structure.bg!75!black}
\setbeamercolor*{example text}{fg=green!65!black}
\setbeamercolor*{block body}{bg=structure.bg!90!black}
\setbeamercolor*{block body alerted}{bg=structure.bg!90!black}
\setbeamercolor*{block body example}{bg=structure.bg!90!black}
\setbeamercolor*{block title}{parent=structure,bg=structure.bg!75!black}
\setbeamercolor*{block title alerted}{use={structure,alerted text},fg=alerted text.fg!75!structure.fg,bg=structure.bg!75!black}
\setbeamertemplate{navigation symbols}{}
\setbeamertemplate{items}[square]
\setbeamercolor{item projected}{fg=white}
\setbeamercolor*{normal text}{fg=white!90!blue,bg=blue!70!black}
\setbeamercolor*{separation line}{}
\setbeamercolor*{fine separation line}{}
\setbeamercolor{alerted text}{fg=green}

\usepackage[italian]{babel}
\usepackage[utf8]{inputenc}
\usepackage{pgf}
\usepackage{verbatim}
\usepackage{inconsolata}
\usepackage{listings}
\lstset{language=C, frame=single,
  basicstyle=\ttfamily,
  numbers=left, numberstyle=\tiny\color{gray},
  numbersep=5pt, fancyvrb=true
}

\usefonttheme{professionalfonts} % using non standard fonts for beamer
\usefonttheme{serif} % default family is serif
% \usepackage{fontspec}
% \setmainfont{Palatino}

\usepackage{pgf}
\usepackage{tikz}
\usepackage{graphicx}
\usetikzlibrary{%
  arrows,
  arrows.meta,
  positioning,
  calc,
  backgrounds,
  chains,
  matrix,
  patterns,
  automata,
  fit,
  graphs,
  decorations,
  decorations.pathmorphing,
  decorations.pathreplacing,
  decorations.markings,
}

\usepgflibrary{shapes,shapes.geometric}


\usepackage{url}
\usepackage{xmpmulti}
% \usepackage{euler}
%\usepackage[T1]{fontenc}
\pdfpagebox5
% \immediate\write18{sh ./vc}
% \input{vc}

\author{Gianluca Della Vedova}
\title{Elementi di Bioinformatica}
\institute{Univ. Milano--Bicocca\\
  \texttt{https://gianluca.dellavedova.org}}
\date{\today}
%\pgfdeclareimage[height=1cm]{university-logo}{logounimib}
%\logo{\pgfuseimage{university-logo}}



\begin{document}
\begin{frame}
      \titlepage

        \centering
          Allineamento 
\end{frame}


\begin{frame}[fragile]
\frametitle{Allineamento di 2 sequenze}
\begin{block}{Allineamento}
\begin{itemize}
\item
Input: 2 sequenze $s_{1}$ e $s_{2}$
\item
Aggiunta di \alert{indel} in $s_{1}$ e $s_{2}$
\item
sequenze estese = stessa lunghezza
\item
NO colonne di indel
\end{itemize}
\end{block}
\end{frame}

\begin{frame}[fragile]
\frametitle{Allineamento: esempio}
\begin{block}{Input}
\texttt{ABRACADABRA}\\
\texttt{BANANA}
\end{block}

\begin{block}{sequenze allineate 1}
\texttt{ABRACADABRA}\\
\texttt{-B-ANA---NA}
\end{block}
\begin{block}{sequenze allineate 2}
\texttt{ABR-AC-ADABRA}\\
\texttt{---B-ANA---NA}
\end{block}
\begin{block}{sequenze allineate 3}
\texttt{ABRACADABRA}\\
\texttt{-BANA----NA}
\end{block}
\end{frame}

\begin{frame}[fragile]
\frametitle{Allineamento: costo o valore?}
\begin{block}{Problema di ottimizzazione}
\begin{itemize}
\item
Istanza: insieme infinito di casi
\item
Soluzioni ammissibili: ammissibilità verificabile in tempo polinomiale
\item
Funzione obiettivo: soluzione ammissibile $\mapsto \mathbb{Q}$
\item
Massimizzazione o minimizzazione
\end{itemize}
\end{block}
\begin{block}{costo o valore?}
\begin{itemize}
\item
Costo da minimizzare
\item
Valore da massimizzare
\end{itemize}
\end{block}
\end{frame}

\begin{frame}[fragile]
\frametitle{Valore}
\begin{block}{Valore di un allineamento}
\begin{itemize}[<+->]
\item
Somma dei valori delle singole colonne
\item
Valore di una colonna =
\item
valore in ingresso
\end{itemize}
\end{block}

\only<4->{\begin{block}{Istanza}
\begin{itemize}
\item
due sequenze $s_{1}$ e $s_{2}$
\item
matrice di score
$d:\left(\Sigma\cup\{-\}\right)\times\left(\Sigma\cup\{-\}\right) \mapsto
\mathbb{Q}$ (normalmente interi)
\item
problema di massimizzazione = massima omologia
\end{itemize}
\end{block}}
\end{frame}

\begin{frame}[fragile]
\frametitle{Needleman-Wunsch: Equazione di ricorrenza}
\begin{block}{Definizione}
$M[i,j] = $ ottimo su $s_{1}[:i]$, $s_{2}[:j]$
\end{block}
\begin{equation*}
M[i,j] = \max \left\{
\begin{array}{r}%{r@{\quad }l}
M[i-1, j-1] + d(s_{1}[i], s_{2}[j])\\
M[i  , j-1] + d( -     , s_{2}[j])\\
M[i-1, j  ] + d(s_{1}[i], -      )
\end{array}
\right.
\end{equation*}
\begin{block}{Condizione al contorno}
\begin{itemize}
\item
$M[0,0] = 0$
\item
$M[i,0] = M[i-1,0] + d(s_{1}[i],-)$
\item
$M[0,j] = M[0,j-1 + d(-,s_{2}[j])$
\end{itemize}
\end{block}
\end{frame}


\begin{frame}[fragile]
\frametitle{Allineamento locale}
\begin{block}{Allineamento globale}
Si allineano le sequenze intere
\end{block}
\begin{block}{Allineamento locale}
\begin{enumerate}
\item
Input: $s_{1}$, $s_{2}$, matrice di score $d$
\item
Individuare sottostringhe $t_{1}$ di $s_{1}$ e $t_{2}$ di $s_{2}$ tale che
\item
$All[t_{1}, t_{2}] \geq All[u_{1}, u_{2}]$ per ogni coppia di sottostringhe
$u_{1}, u_{2}$ di $s_{1}, s_{2}$.
\item
Algoritmo banale:
calcolo tutte le sottostringhe di $s_{1}$, $s_{2}$ e ne
calcolo allineamento globale
\item
Tempo $O(n^{3}m^{3})$
\end{enumerate}
\end{block}
\end{frame}

\begin{frame}[fragile]
\frametitle{Smith-Waterman}
\begin{block}{Osservazione 1}
\begin{enumerate}
\item
Matrice $M[i,j]$ memorizza allineamento di tutte le coppie di \alert{prefissi} di
$s_{1}$, $s_{2}$
\item
Allineamento massimo fra coppie di prefissi = valore massimo in $M$
\end{enumerate}
\end{block}
\begin{block}{Osservazione 2}
\begin{enumerate}
\item
$M[0,0] = 0$
\item
quindi non si prendono sottostringhe con allineamento negativo
\end{enumerate}
\end{block}
\end{frame}

\begin{frame}[fragile]
\frametitle{Equazione di ricorrenza}
\begin{block}{Definizione}
$M[i,j] = $ ottimo fra tutte le stringhe $s_{1}[k:i]$, $s_{2}[h:j]$
\end{block}
\begin{equation*}
M[i,j] = \max \left\{
\begin{array}{r}%{r@{\quad }l}
M[i-1, j-1] + d(s_{1}[i], s_{2}[j])\\
M[i  , j-1] + d( -     , s_{2}[j])\\
M[i-1, j  ] + d(s_{1}[i], -      )\\
0
\end{array}
\right.
\end{equation*}
\begin{block}{Condizione al contorno}
\begin{itemize}
\item
$M[0,0] = M[i,0] = M[0,j] =0$
\item
punto finale = valore massimo
\item
si risale nell'allineamento fino a uno $0$.
\item
Tempo $(nm)$
\end{itemize}
\end{block}
\end{frame}



\begin{frame}[fragile]
\frametitle{Gap}
\begin{block}{Definizione}
\begin{enumerate}
\item
Sequenza contigua di indel in un \alert{allineamento}
\end{enumerate}
\end{block}
\begin{block}{Esempio}
\texttt{ABR\alert{-}AC\alert{-}ADABRA}: 2 gap\\
\texttt{\alert{---}B\alert{-}ANA\alert{---}NA}: 3 gap
\end{block}
\begin{block}{Osservazione}
\begin{enumerate}
\item
Un gap sposta il frame di lettura
\item
no indel $\ne$ 1 indel
\\item
1 indel $\approx$ 2 indel
\end{enumerate}
\end{block}
\end{frame}

\begin{frame}[fragile]
\frametitle{Allineamento con gap generici}
\begin{itemize}[<+->]
\item
costo gap lungo $l$: $P(l)$
\item
Come descrivo l'allineamento ottimo?
\item
Come è fatta l'ultima colonna?
\item
Come è fatto l'ultimo gap?
\end{itemize}
\end{frame}

\begin{frame}[fragile]
\frametitle{Gap generico}
\begin{block}{Definizione}
$M[i,j] = $ ottimo su $s_{1}[:i]$, $s_{2}[:j]$
\end{block}
\begin{equation*}
M[i,j] = \max \left\{
\begin{array}{r}%{r@{\quad }l}
M[i-1, j-1] + d(s_{1}[i], s_{2}[j])\only<2>{\alert{\text{ no gap}}}\\
\max_{l>0} M[i  , j-l] + P(l)\only<2>{\alert{\text{ gap in }s_{1}}}\\
\max_{l>0} M[i-l, j  ] + P(l)\only<2>{\alert{\text{ gap in }s_{2}}}
\end{array}
\right.
\end{equation*}
\begin{block}{Condizione al contorno}
\begin{itemize}[<+->]
\item
$M[0,0] = 0$
\item
$M[i,0] = P(i)$, $M[0,j] = P(j)$
\item
Tempo $O(nm(n+m))$
\end{itemize}
\end{block}
\end{frame}


\begin{frame}[fragile]
\frametitle{Allineamento con gap affine}
\begin{itemize}[<+->]
\item
costo gap lungo $l$: $P_{o} + lP_{e}$
\item
$P_{o}$: costo apertura gap
\item
$P_{e}$: costo estensione gap
\item
$P_{e}, P_{o}>0$
\item
Come descrivo l'allineamento ottimo?
\item
Come è fatta l'ultima colonna?
\item
Come è fatto l'ultimo gap?
\end{itemize}
\end{frame}


\begin{frame}[fragile]
\frametitle{Gap affine}
\begin{block}{Definizione}
\begin{itemize}
\item
$M[i,j] = $ ottimo su $s_{1}[:i]$, $s_{2}[:j]$
\item
$E_{1}[i,j] = $ ottimo su $s_{1}[:i]$, $s_{2}[:j]$, con estensione di gap finale
in $s_{1}$
\item
$E_{2}[i,j] = $ ottimo su $s_{1}[:i]$, $s_{2}[:j]$, con estensione di gap finale
in $s_{2}$
\item
$N_{1}[i,j] = $ ottimo su $s_{1}[:i]$, $s_{2}[:j]$, con apertura di gap alla
fine di $s_{1}$
\item
$N_{2}[i,j] = $ ottimo su $s_{1}[:i]$, $s_{2}[:j]$, con apertura di gap alla
fine di $s_{2}$
\end{itemize}
\end{block}
\end{frame}

\begin{frame}[fragile]
\frametitle{Gap affine}
\begin{equation*}
M[i,j] = \max \left\{
\begin{array}{l}%{r@{\quad }l}
M[i-1, j-1] + d(s_{1}[i], s_{2}[j])\\
E_{1}[i,j],
E_{2}[i,j] \\
N_{1}[i,j],
N_{2}[i,j]
\end{array}
\right.
\end{equation*}
\begin{equation*}
E_{1}[i,j] = \max \left\{
\begin{array}{l}%{r@{\quad }l}
E_{1}[i,j-1] + P_{e}\\
N_{1}[i,j-1] + P_{e}
\end{array}
\right.
\end{equation*}
\begin{equation*}
E_{2}[i,j] = \max \left\{
\begin{array}{l}%{r@{\quad }l}
E_{2}[i-1,j] + P_{e} \\
N_{2}[i-1,j] + P_{e}
\end{array}
\right.
\end{equation*}
\begin{equation*}
N_{1}[i,j] = M[i,j-1] + P_{o} + P_{e}, \quad
N_{2}[i,j] = M[i-1,j] + P_{o} + P_{e}
\end{equation*}
\end{frame}

\begin{frame}[fragile]
\frametitle{Allineamento multiplo}
\begin{block}{$k$ sequenze}
\begin{itemize}[<+->]
\item
Input: insieme di sequenze $\{s_{1}, \ldots , s_{k}\}$
\item
Aggiunta di \alert{indel} nelle sequenze
\item
sequenze estese = tutte stessa lunghezza
\item
NO colonne di indel
\end{itemize}
\end{block}
\end{frame}

\begin{frame}[fragile]
\frametitle{Valore di una colonna}
\begin{block}{SP: sum of pairs}
\begin{itemize}[<+->]
\item
$\{s_{1}, \ldots , s_{k}\} \mapsto \{s^{*}_{1}, \ldots , s^{*}_{k}\}$ allineate
\item
Valore $\{s^{*}_{1}[h], \ldots , s^{*}_{k}[h]\}$
\item
$\sum_{i<j} d(s^{*}_{1}[i], s^{*}_{k}[j])$
\end{itemize}
\end{block}
\begin{block}{Complessità}
\begin{itemize}[<+->]
\item
se $k$ è arbitrario $\Rightarrow$ NP-completo
\item
se $k$ è fissato $\Rightarrow$ tempo $O(n^{k})$
\end{itemize}
\end{block}
\end{frame}


\begin{comment}
\begin{frame}[fragile]
\frametitle{Allineamento con banda.}
\begin{enumerate}
\item
$Opt$ = valore allineamento
\end{frame}
\end{frame}

\begin{comment}
\begin{frame}[fragile]
\frametitle{Algoritmo di Carrillo-Lipman}
\begin{quote}{Euristica}
\begin{enumerate}
\item
Input: $3$ sequenze $s_{1}$, $s_{2}$, $s_{3}$
\item
Non visitiamo tutta la matrice di programmazione dinamica
\item
Caso peggiore $O(n^{3})$
\end{enumerate}
\end{quote}
\end{frame}
\end{comment}

\begin{frame}[fragile]
\frametitle{Matrici di sostituzione}
\begin{enumerate}
\item
Utilizzate per valutare un allineamento
\item
Implicitamente probabilità di transizione
\item
Mutazioni ricorrenti
\item
Allineamenti di proteine
\end{enumerate}
\end{frame}

\begin{frame}[fragile]
\frametitle{PAM: unità di misura}
\begin{enumerate}
\item
PAM: point/percent accepted mutation
\item
due sequenze $s_{1}$ e $s_{2}$: quanto sono distanti?
\item
distanza 1PAM $\Rightarrow$ numero mutazioni = $\frac{1}{100} |s_{1}|$
\item
semplice in assenza di indel
\item
Mutazioni ricorrenti $\Rightarrow$ misura affidabile solo per piccoli valori
\item
$s_{1}$ e $s_{2}$ distanti $100$ PAM $\Rightarrow$ una singola base ha 36\% di
probabilità di non essere mutata
\end{enumerate}
\end{frame}

\begin{frame}[fragile]
\frametitle{Matrici PAM}
\begin{enumerate}
\item
dipende dalla distanza attesa
\item
PAM250, PAM200, PAM1
\end{enumerate}
\begin{block}{Calcolo PAM$k$}
\begin{enumerate}
\item
Costruzione PAM$k$
\item
Si prendono varie sequenze distanti $k$PAM
\item
si allineano le sequenze
\item
si calcolano le frequenze $f(i)$, $f(i,j)$ di tutti i singoli caratteri e le
coppie di caratteri
\item
PAM$k(i,j)=\log\frac{f(i,j)}{f(i)f(j)}$
\end{enumerate}
\end{block}
\end{frame}

\begin{frame}[fragile]
\frametitle{Log odds ratio}
\begin{block}{Odds ratio}
\begin{enumerate}
\item
$\frac{p}{1-p}$, $p$ è la probabilità dell'evento interessante (target)
\item
$\frac{f(i,j)}{f(i)f(j)}$
\item
$f(i,j)$: frequenza della mutazione misurata
\item
$f(i)f(j)$: ipotesi nulla (caratteri indipendenti)
\end{enumerate}
\end{block}
\end{frame}


\begin{frame}[fragile]
\frametitle{Matrici PAM}
\begin{block}{Calcolo PAM$k$ nella realtà}
\begin{itemize}
\item
Problema: come allineare se non si conosce la matrice
\item
Allineate sequenze molto simili
\item
no indel
\item
$M_{k}(i,j)=\log\frac{f(i)M_{1}^{k}(i,j)}{f(i)f(j)}=\log\frac{M_{1}^{k}(i,j)}{f(j)}$
\item
valori moltiplicati per $10$
\item
arrotondati all'intero più vicino
\item
si somma un intero a tutti i valori
\end{itemize}
\end{block}
\end{frame}


\begin{frame}[fragile]
\frametitle{Matrici BLOSUM}
\begin{block}{Confronto con PAM}
\begin{itemize}
\item
PAM allinea sequenze vicine
\item
ma viene usata per allineare sequenze lontane
\item
regioni conservate e non conservate hanno stessa importanza
\end{itemize}
\end{block}

\begin{block}{BLOCKS}
\begin{itemize}
\item
blocchi di regioni conservate
\item
scelte ``a mano''
\item
$B(i,j)=\log\frac{f(i,j)}{f(i)f(j)}$
\end{itemize}
\end{block}
\end{frame}

\begin{frame}[fragile]
\frametitle{Matrici BLOSUM}
\begin{block}{BLOSUM$x$}
\begin{itemize}
\item
le sequenze che sono simili più di $x$\% vengono clusterizzate
\item
cluster = rimuovere tutte tranne una
\item
scopo:  evitare di sovrapesare parti sovrarappresentate nel campione
\item
BLOSUM62: più usata per gli allineamenti
\end{itemize}
\end{block}
\end{frame}


\begin{frame}[fragile]
\frametitle{Statistiche Karlin-Altschul}
\begin{block}{Ricerca in un database}
\begin{itemize}
\item
Punteggio positivo possibile
\item
Punteggio medio negativo
\item
Simboli indipendenti e equiprobabili
\item
Sequenze infinitamente lunghe
\item
Allineamenti senza gap
\end{itemize}
\end{block}
\end{frame}

\begin{frame}[fragile]
\frametitle{Equazione Karlin-Altschul}
\begin{equation*}
E=kmne^{-\lambda S}
\end{equation*}
\begin{itemize}
\item
$E$: numero allineamenti
\item
$k$: costante
\item
$n$: numero caratteri in database
\item
$m$: lunghezza stringa query
\item
$\lambda S$: punteggio normalizzato
\end{itemize}
\end{frame}

\begin{frame}[fragile]
\frametitle{BLAST}
\begin{block}{Basic Local Alignment Search Tool}
\begin{itemize}
\item
Ricerca seed
\item
seed = pattern matching con sottostringa di lunghezza $3$
\item
Costruzione  high-scoring segment pair (HSP)
= estensione seed
\item
Filtro seed tenuti solo HSP con  alta significatività
\item
Fusione HSP  vicine
\item
Smith-Waterman sulle regioni
\end{itemize}
\end{block}
\end{frame}


\begin{frame}[containsverbatim]\frametitle{Licenza d'uso}
  \small

  Quest'opera {\`e} soggetta alla licenza Creative Commons:
Attribuzione-Condividi allo stesso modo 4.0.
  (\verb+https://creativecommons.org/licenses/by-sa/4.0/+).

Sei libero di riprodurre, distribuire, comunicare al pubblico, esporre
in pubblico, rappresentare, eseguire, recitare e modificare quest'opera
alle seguenti condizioni:
\begin{itemize}
\item
Attribuzione — Devi attribuire la paternit{\`a} dell'opera nei modi
indicati dall'autore o da chi ti ha dato l'opera in licenza e in modo tale da
non suggerire che essi avallino te o il modo in cui tu usi l'opera.
\item
Condividi allo stesso modo — Se alteri o trasformi quest'opera, o se
la usi per crearne un'altra, puoi distribuire l'opera risultante solo con
una licenza identica o equivalente a  questa.
\end{itemize}
%  \vspace*{1cm}
\end{frame}

\end{document}
